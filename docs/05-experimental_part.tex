\chapter{Исследовательская часть}
Цель исследования - сравнить зависимость времени выдачи наград от количества вознаграждаемых игроков в двух случаях: при помощи хранимой процедуры и при помощи обычных
запросов.

Описание способов выдачи наград:
\begin{itemize}
	\item хранимая процедура~---~вызов хранимой процедуры (один запрос на СУБД);
	\item обычные запросы~---~реализация логики добавления наград с помощью EF Core (множество запросов в СУБД).
\end{itemize}

Замеры времени нуждаются в нескольких вариантах изначальных входных данных, что требует создание независимых вариантов базы данных, инициализируемые различными параметрами. Для создания независимых образов PostgreSQL использовалась библиотека Testcontainers.Postgresql~\cite{testcontainers}

Исследование проводилось в два этапа: подготовка данных и замеры времени.

На каждую подготовку данных выделялся отдельный образ PostgreSQL, которым был инициализирован сценарием (?). Программа заполняла СУБД нужными данными, после чего делала снимок базы данных при помощи pg\_dump~\cite{postgresql_pgdump} в файл. Для заполнения базы данных случайными данными использовалась библиотека Bogus~\cite{bogus}.

Если снимок данных подготавливался для $N$ выдаваемых наград, генерировалось:
\begin{itemize}
	\item 1 соревнование;
	\item $N/2$ участников и записей в таблицы лидеров;
	\item 5 критериев наград, каждый из которых вознаграждает $\frac{N}{5}$ участников.
\end{itemize}
Критерии выдачи награды генерировались с равной вероятностью двух возможных видов: по диапазону доли в таблице лидеров, по диапазону мест.

На каждый замер времени выделялся отдельный образ PostgreSQL, который был инициализирован снимком базы данных, полученным на этапе подготовке данных. После этого, выполнялись замеры времени работы или хранимой процедуры, или обычных запросов. Для измерения реального времени работы использовался класс Stopwatch~\cite{stopwatch}



\section*{Вывод}

TODO

\clearpage
