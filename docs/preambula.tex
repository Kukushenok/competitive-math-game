\documentclass[a4paper, 12pt]{extreport}

\usepackage[T2A]{fontenc}
\usepackage[utf8]{inputenc}
\usepackage[english,russian]{babel}
\usepackage{amssymb,amsfonts,amsmath,mathtext,enumerate,float} %cite is removed for biblatex
\usepackage{pgfplots}
\usepackage{graphicx}
\usepackage{tocloft}
\usepackage{listings}
\usepackage{caption}
\usepackage{tempora}
\usepackage{titlesec}
\usepackage{setspace}
\usepackage{geometry}
\usepackage{indentfirst}
\usepackage{pdfpages}
\usepackage{enumerate,letltxmacro}
\usepackage{threeparttable}
\usepackage{hyperref}
\usepackage{flafter}
\usepackage{enumitem}
\usepackage{multirow}
\usepackage{csvsimple}
\usepackage[figure,table]{totalcount}
\usepackage{lastpage}
\usepackage{pdfpages}
\usepackage{longtable}
\usepackage{placeins}
\usepackage{siunitx}
\usepackage{array}
\usepackage[figure,table]{totalcount}
\usepackage{totcount}
\newtotcounter{citenum}
\newtotcounter{imgnum}
\usepackage[
	style=gost-numeric,
	language=auto,
	autolang=other,
	sorting=none,
]{biblatex}
\setlist{nosep}

\newcommand{\ssr}[1]{\begin{center}
		\LARGE\bfseries{#1}
	\end{center} \addcontentsline{toc}{chapter}{#1}  }

\makeatletter
\renewcommand\LARGE{\@setfontsize\LARGE{22pt}{20}}
\renewcommand\Large{\@setfontsize\Large{20pt}{20}}
\renewcommand\large{\@setfontsize\large{16pt}{20}}
\makeatother
\RequirePackage{titlesec}
\titleformat{\chapter}[block]{\hspace{\parindent}\large\bfseries}{\thechapter}{0.5em}{\large\bfseries\raggedright}
\titleformat{name=\chapter,numberless}[block]{\hspace{\parindent}}{}{0pt}{\large\bfseries\centering}
\titleformat{\section}[block]{\hspace{\parindent}\large\bfseries}{\thesection}{0.5em}{\large\bfseries\raggedright}
\titleformat{\subsection}[block]{\hspace{\parindent}\large\bfseries}{\thesubsection}{0.5em}{\large\bfseries\raggedright}
\titleformat{\subsubsection}[block]{\hspace{\parindent}\large\bfseries}{\thesubsection}{0.5em}{\large\bfseries\raggedright}
\titlespacing{\chapter}{12.5mm}{-22pt}{10pt}
\titlespacing{\section}{12.5mm}{10pt}{10pt}
\titlespacing{\subsection}{12.5mm}{10pt}{10pt}
\titlespacing{\subsubsection}{12.5mm}{10pt}{10pt}

\makeatletter
\renewcommand{\@biblabel}[1]{#1.}
\makeatother
%
%\titleformat{\chapter}[hang]{\LARGE\bfseries}{\hspace{1.25cm}\thechapter}{1ex}{\LARGE\bfseries}
%\titleformat{\section}[hang]{\Large\bfseries}{\hspace{1.25cm}\thesection}{1ex}{\Large\bfseries}
%\titleformat{name=\section,numberless}[hang]{\Large\bfseries}{\hspace{1.25cm}}{0pt}{\Large\bfseries}
%\titleformat{\subsection}[hang]{\large\bfseries}{\hspace{1.25cm}\thesubsection}{1ex}{\large\bfseries}
%\titlespacing{\chapter}{0pt}{-\baselineskip}{\baselineskip}
%\titlespacing*{\section}{0pt}{\baselineskip}{\baselineskip}
%\titlespacing*{\subsection}{0pt}{\baselineskip}{\baselineskip}

\geometry{left=30mm}
\geometry{right=10mm}
\geometry{top=20mm}
\geometry{bottom=20mm}

\onehalfspacing

\renewcommand{\theenumi}{\arabic{enumi}}
\renewcommand{\labelenumi}{\arabic{enumi}\text{)}}
\renewcommand{\theenumii}{.\arabic{enumii}}
\renewcommand{\labelenumii}{\asbuk{enumii}\text{)}}
\renewcommand{\theenumiii}{.\arabic{enumiii}}
\renewcommand{\labelenumiii}{\arabic{enumi}.\arabic{enumii}.\arabic{enumiii}.}

\renewcommand{\cftchapleader}{\cftdotfill{\cftdotsep}}

\addto\captionsrussian{\renewcommand{\figurename}{Рисунок}}
\DeclareCaptionLabelSeparator{dash}{~---~}
\captionsetup{labelsep=dash}

\captionsetup[figure]{justification=centering,labelsep=dash}
\captionsetup[table]{labelsep=dash,justification=raggedright,singlelinecheck=off}

\graphicspath{{images/}}%путь к рисункам

\newcommand{\floor}[1]{\lfloor #1 \rfloor}

\lstset{ %
	language=caml,                 % выбор языка для подсветки (здесь это С)
	basicstyle=\small\ttfamily, % размер и начертание шрифта для подсветки кода
	numbers=none,               % где поставить нумерацию строк (слева\справа)
	numberstyle=\tiny,           % размер шрифта для номеров строк
	stepnumber=1,                   % размер шага между двумя номерами строк
%	numbersep=5pt,                % как далеко отстоят номера строк от подсвечиваемого кода
	showspaces=false,            % показывать или нет пробелы специальными отступами
	showstringspaces=false,      % показывать или нет пробелы в строках
	showtabs=false,             % показывать или нет табуляцию в строках
	frame=single,              % рисовать рамку вокруг кода
	tabsize=2,                 % размер табуляции по умолчанию равен 2 пробелам
	captionpos=t,              % позиция заголовка вверху [t] или внизу [b] 
	breaklines=true,           % автоматически переносить строки (да\нет)
	breakatwhitespace=false, % переносить строки только если есть пробел
	escapeinside={\#*}{*)},   % если нужно добавить комментарии в коде
	abovecaptionskip=-5pt
}

\pgfplotsset{width=0.85\linewidth, height=0.5\columnwidth}

\linespread{1.3}

\parindent=1.25cm

%\LetLtxMacro\itemold\item
%\renewcommand{\item}{\itemindent0.75cm\itemold}

\def\labelitemi{---}
\setlist[itemize]{leftmargin=1.25cm, itemindent=0.65cm}
\setlist[enumerate]{leftmargin=1.25cm, itemindent=0.55cm}

% My additions :3

\NewBibliographyString{accessmode}
\DeclareFieldFormat{url}{\bibstring{accessmode}\addcolon\space\url{#1}}
\DeclareFieldFormat{title}{#1}
\DefineBibliographyStrings{russian}{
	accessmode={Режим доступа:},
	urlseen={дата обращения:}
}
\AtDataInput{\stepcounter{citenum}}

\newcommand{\img}[3] {
	\stepcounter{imgnum}
	\begin{figure}[h!]
		\center{\includegraphics[height=#1]{images/images/#2}}
		\caption{#3}
		\label{img:#2}
	\end{figure}
}

\newcommand{\scr}[2] {
	\stepcounter{imgnum}
	\begin{figure}[!ht]
		\center{\includegraphics[width=\textwidth]{images/screenshots/#1}}
		\caption{#2}
		\label{scr:#1}
	\end{figure}
}


\newcommand{\scrw}[3] {
	\stepcounter{imgnum}
	\begin{figure}[!ht]
		\center{\includegraphics[width=#1\textwidth]{images/screenshots/#2}}
		\caption{#3}
		\label{scr:#2}
	\end{figure}
}

\newcommand{\demoimg}[2] {
	\stepcounter{imgnum}
	\begin{figure}[h!]
		\centering
		\includegraphics[width=1.0\textwidth]{images/images/#1}
		\caption{#2}
		\label{img:#1}
	\end{figure}
}


\newcommand{\scheme}[2] {
	\stepcounter{imgnum}
	\begin{figure}[H]
		\centering
		\includegraphics[width=1.0\textwidth]{images/schemes/#1.pdf}
		\caption{#2}
		\label{scheme:#1}
	\end{figure}
}

\newcommand{\graph}[2] {
	\stepcounter{imgnum}
	\begin{figure}[H]
		\centering
		\includegraphics[width=1.0\textwidth]{images/plots/#1.pdf}
		\caption{#2}
		\label{graph:#1}
	\end{figure}
}


\usepackage{listings}
\usepackage{xcolor}
\lstdefinestyle{code}{
	language=sql,
	backgroundcolor=\color{white},
	basicstyle=\footnotesize\ttfamily,
	keywordstyle=\color{black},
	stringstyle=\color{black},
	commentstyle=\color{black},
	directivestyle=\color{black},
	numbers=left,
	numberstyle=\tiny,
	stepnumber=1,
	numbersep=5pt,
	frame=single,
	tabsize=4,
	captionpos=t,
	breaklines=true,
	breakatwhitespace=true,
	escapeinside={\#*}{*)},
	morecomment=[l][\color{magenta}]{\#},
	columns=fullflexible,
	xleftmargin=10pt
}
\newcommand{\code}[2]{
	\begin{lstinputlisting}[
		caption={#2},
		label={code:#1},
	    style={code}
	]{code/#1}
	\end{lstinputlisting}
}

\newcommand{\specialcell}[2][c]{%
	\begin{tabular}[#1]{@{}c@{}}#2\end{tabular}}

\frenchspacing
