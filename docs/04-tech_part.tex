\chapter{Технологическая часть}

В качестве реляционной СУБД была выбрана PostgreSQL~\cite{postgresql}, поскольку в ней представлены все необходимые возможности для создания реляционной базы данных, в частности создание хранимых процедур и ролевой системы.

Для реализации был выбран язык C\# версии 12, поскольку в нём представлены все необходимые возможности для создания Web-интерфейса взаймодействия с приложением, в частности, с помощью фреймворка ASP.NET Core~\cite{aspnet}.

Для реализации внедрения зависимостей использовалась библиотека Microsoft.Dependency-Injection~\cite{di}

Для связи объектов базы данных с кодом программы использовался фреймворк EF Core~\cite{efcore}.

Для обработки изображений была выбрана библиотека Magick.NET~\cite{magicknet}. 

Для хранения и выполнения отложенных задач использовалась библиотека Quartz.NET~\cite{quartznet}.

\section{Компоненты программного обеспечения}
На рисунке~\ref{scheme:components} представлена диаграмма компонентов приложения.
\scheme{components}{Диаграмма компонентов приложения}

Модуль Core содержит определение базовых объектов в виде классов, с которыми работает компонент доступа к базе данных и компонент бизнес-логики.

Модуль Repositories содержит интерфейсы, определяющие взаимодействия с базой данных. 

Модуль RepositoriesRealisation содержит реализацию уровня взаимодействия с базой данных. 

Модуль Services содержит интерфейсы, определяющие уровень бизнес-логики, а также сервиса постановки задач с отложенным ожиданием и сервиса обработки изображений.

Модуль ChronoServiceRealisation содержит реализацию сервиса постановки задач с отложенным ожиданием.

Модуль ImageProcessorRealisation содержит реализацию сервиса обработки изображений.

Модуль ServicesRealisation содержит реализацию уровня бизнес-логики на основе интерфейсов компонентов доступа к базе данных.

Модуль BackendUsage содержит интерфейсы, определяющие взаимодействие с приложением в общем виде.

Модуль BaseUsage содержит реализацию взаимодействия с приложением в общем виде на основе интерфейсов уровня бизнес-логики.

Модуль WebBackend реализует сетевое взаимодействие на основе интерфейсов взаимодействия с приложением в общем виде.

Модуль TechnologicalUIHost реализует взаимодействие в виде консольного интерфейса на основе интерфейсов взаимодействия с приложением в общем виде. Также он определяет интерфейс, описывающий консольный ввод-вывод.

Модуль TechnologicalUI содержит реализацию консольного ввода-вывода и предоставляет  консольный интерфейс.

Модуль SolutionInstaller настраивает внедрение зависимостей, сопоставляя всем интерфейсам взаимодействия базы данных, бизнес~--~логики и т.д. с их реализациями.

\section{Тестирование}

Для реализации компонента бизнес~--~логики были разработаны модульные тесты. Для имитирования объектов слоя доступа к базы данных использовалась библиотека Moq~\cite{moq}. Составлены 81~тест, обеспечивающие покрытие кода~---~82.8\%. Все тесты пройдены успешно.

Для реализации компонента взаимодействия с базой данных были разработаны интеграционные тесты. Для создания независимых образов базы данных использовалась библиотека Testcontainers~\cite{testcontainers}. Составлено 108~тестов, обеспечивающие покрытие кода~---~79.42\%. Все тесты пройдены успешно.

Для реализации компонента обработки изображений были разработаны функциональные тесты. На вход компоненту поступают файлы из папки Tests с различными названиями:
\begin{itemize}
	\item если файл имеет префикс neg, то обработка должна выдать исключение; результат не сохраняется.
	\item в противном случае, ожидается, что обработка изображения должна пройти успешно; при этом результат сохраняется в соседнюю папку Results.
\end{itemize}
Составлено 15 тестов, которые обеспечивают 100\% покрытие кода класса обработки изображений. Все тесты пройдены успешно.

Для реализации компонента отложенных задач были разработаны интеграционные тесты. Составлено 7~тестов, которые обеспечивают 100\% покрытие кода класса отложенных задач. Все тесты пройдены успешно.

\section{Примеры работы}
На рисунках~\ref{scr:sc1}~--~\ref{scr:sc4} приведены примеры работы программы, используя Web~---~интерфейс.

\scr{sc1}{Демонстрация регистрации в приложении}
\scr{sc2}{Демонстрация изменение профиля}
\scr{sc3}{Демонстрация участия в соревновании}
\scr{sc4}{Демонстрация просмотра таблицы лидеров}
\FloatBarrier

\section*{Вывод}

Были выбраны средства реализации базы данных и приложения. Реализованы все компоненты Описаны методы тестирования разработанного функционала и разработать тесты для проверки корректности работы приложения. Все тесты пройдены успешно.

\clearpage
