\chapter{Технологическая часть}

В качестве СУБД была выбрана реляционная система управления базами данных PostgreSQL~\cite{postgresql}, поскольку в ней представлены все необходимые возможности, в частности создание хранимых процедур.

Для реализации приложения был выбран язык C\# версии 11.0 вместе с фреймворком ASP.NET Core, поскольку в нём представлены все необходимые возможности для создания программного интерфейса взаимодействия с базой данных.

Для интерфейса доступа к СУБД был выбран Entity Framework Core, поскольку он предоставляет возможность взаимодействия с объектами базы данных как с классами C\#.

Для обработки изображений была выбрана библиотека Magick.NET~\cite{magicknet}. 

Для отложенных задач использовалась библиотека Quartz.NET~\cite{quartznet}.

\section{Модули}
На рисунке~\ref{scheme:components} представлена диаграмма компонентов приложения.
\scheme{components}{Диаграмма компонентов приложения}

Модуль Core содержит описание базовых объектов, с которыми работает приложение.

Модуль Repositories содержит описание уровня взаимодействия с базой данных. 

Модуль RepositoriesRealisation содержит реализацию уровня взаимодействия с базой данных. 

\section{Тестирование}

Для тестирования приложения были составлены следующие виды тестов:
\begin{itemize}
	\item интеграционные тесты для компонентов реализации взаимодействия с базой данных;
	\item интеграционные тесты для реализации компонента отложенных задач;
	\item интеграционные тесты для реализации компонента обработки изображений;
	\item модульные тесты для реализации компонента бизнес~--~логики.
\end{itemize}

\section*{Вывод}

TODO

\clearpage
