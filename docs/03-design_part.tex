\chapter{Конструкторская часть}

%Программное обеспечение функционально должно обеспечивать

%Программное обеспечение должна предоставлять пользователю следующие возможности
\section{Проектируемая база данных}

\scheme{db}{Схема проектируемой базы данных}


\section{Функциональная модель}
При истечении срока соревнования, игрокам должны быть выданы награды, исходя из таблицы лидеров. Награды соревнования выдаются всем участникам в таблице лидеров, которые подошли под критерий выдачи награды. После этого, для соревнования устанавливается, что награды за него выданы, чтобы избежать повторных вознаграждений.

Схема хранимой процедуры выдачи наград представлена на рисунках~\ref{scheme:procedure_1},~\ref{scheme:procedure_2}.
\scheme{procedure_1}{Хранимая процедура выдачи наград}
\scheme{procedure_2}{Хранимая процедура выдачи наград}

Для обеспечения безопасности, значение полей "хэш пароля" у таблицы аккаунтов должно быть недоступно. Для этого нужно определить образ таблицы, где данное поле заменено пустыми значениями. Помимо этого, следует определить функцию на уровне базы данных, которая сравнивает хэши паролей.


\section{Ролевая модель}


В рамках задачи было выделено четыре роли на уровне базы данных:
\begin{itemize}
	\item гость,
	\item игрок,
	\item админинстратор,
	\item демон выдачи наград.
\end{itemize}
Гость имеет следующие возможности
\begin{itemize}
	\item читать данные из таблиц
	\item добавлять новые данные в таблицы
	\item вызвать хранимую процедуру
\end{itemize}

\section*{Вывод}

TODO
\clearpage
