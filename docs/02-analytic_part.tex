\chapter{Аналитическая часть}

\section{Формализация соревновательной игры}


Соревновательная игра состоит из основной части, оцениваемая количеством очков (целым числом), и заключительной, где очки игроков сравниваются между собой.

Соревнование характеризуется сроками проведения (начало и конец), а также информацией о выдаваемых наградах. Выдаваемые награды характеризуются собственно наградой и критерием выдачи исходя из таблицы лидеров. Возможны следующие критерии:
\begin{itemize}
	\item место в таблице лидеров~---~игрок занял место в таблице лидеров в пределе указанного диапазона;
	\item ранг в таблице лидеров~---~доля игроков (число от 0 до 1), которых игрок опередил в таблице лидеров, лежит в пределе указанного диапазона.
\end{itemize}
Поскольку формат математической игры подвержен изменению, параметры игры представляют собой бинарный файл, содержащий всю необходимую информацию для расшифровки клиентским приложением. При этом, файл уровня может различаться в зависимости от платформы, на котором установлено клиентское приложение.


%Параметры игры представляют собой текстовый файл, в котором описана нужная для приложения информация об уровне.

В основной части игрок зарабатывает очки, решая математические примеры на скорость, указанные в параметрах соревнования. Результат, как и время подачи результата сохраняется в таблицу лидеров соответствующего соревнования.

Заключительная часть наступает по истечению срока соревнования. Составляется таблица лидеров, сортируя игроков сначала по убыванию игрового счёта, потом по возрастанию времени подачи последнего результата. Считается, что время подачи последнего результата у двух участников не может совпадать. Награды выдаются игрокам, исходя из таблицы лидеров. 

База данных соревновательной игры создаётся для математической игры~---~уровни содержат информацию о примерах, выдаваемых пользователем клиентским приложением. Клиентское приложение осуществляет подсчёт очков и отправляет результат серверу.

%Математический пример в игре состоит из:
%\begin{itemize}
%	\item операции (сложение, умножение, вычитание, деление)
%	\item первого операнда (целое число)
%	\item второго операнда (целое число)
%	\item результата (целое число)
%	\item ложные результаты (различные целые числа, не равные результату)
%\end{itemize}

%Соревновательная игра состоит из основной части, оцениваемая счётом (вещественным числом), и заключительной, где очки игроков сравниваются между собой.

%В основной части игрок зарабатывает очки, решая математические примеры на скорость. Цель игрока~--~решить определённое количество математических примеров правильно
%Игрок начинает с 15 очками. Каждый ход ему предоставляется математический пример, на который он должен ответить, выбрав между несколькими числами-результатами.
%За правильный ответ игрок получает 2 очка, и счётчик оставшихся примеров уменьшается. За неправильный ответ у игрока отнимается 1 очко, а счётчик оставшихся примеров не меняется. Игрок получает следующий пример через секунду после своего ответа.

%Когда все примеры решены, из общего игрока отнимается затраченное время на игру (в секундах). Полученное число является итоговым счётом.

В рамках поставленной цели требуется разработать базу данных, содержащую информацию о соревнованиях, об игроках, о их участии в соревнованиях, а также информацию о выданных наградах и выдаваемых наградах соревнования. Требуется, чтобы программа автоматически выполняла награждение игроков по истечению сроков соревнования. Приложение не должно позволять игроку участвовать вне сроков действия соревнования. 

Желательно, чтобы награды имели представление в виде изображения. Администратор должен иметь возможность изменять его. При этом, загружаемое изображение размером $W \times H$ должно дополнительно обрабатываться:
\begin{itemize}
	\item преобразовываться к размеру $M \times M$, где $M=\min(N,\max(H,W))$, а $N$~--~константа, задающаяся в конфигурационном файле;
	\item преобразовываться к формату JPEG.
\end{itemize}


\section{Анализ существующих решений}

Для сравнения были выбраны соревновательные и обучающие игры. Составлены следующие критерии:
\begin{itemize}
	\item наличие временных соревнований (К1);
	\item внутриигровые вознаграждения за соревнования (К2);
	\item образовательная направленность (К3);
	\item доступность в РФ (К4).
\end{itemize}

Сравнение представлено в таблице~\ref{tbl:comparison}. 

\begin{table}[h!]
	\centering
	\caption{\label{tbl:comparison}Сравнение существующих решений}
	\begin{tabular}{|l|l|l|l|l|}
		\hline
		& К1 & К2 & К3 & K4\\\hline
		Клавогонки & + & -- & + & + \\\hline
		Математический тренажёр & -- & -- & + & +\\\hline
		Clash Royale  & + & + & -- & --\\\hline
		Предлагаемое решение & + & + & + & + \\\hline
	\end{tabular}
\end{table}
\FloatBarrier

Таким образом, разрабатываемое решение не уступает существующим по выдвинутым критериям.

\section{Формализация данных}
Разрабатываемая база данных должна содержать информацию об игроках, профилей игроков, соревнований, результатах игры, наград. Сущности базы данных представлены в таблице~\ref{tbl:dataop}

\begin{table}[h!]
	\centering
	\caption{\label{tbl:dataop}Описание сущностей базы данных}
	\begin{tabular}{|l|p{8cm}|}
		\hline
		Сущность & Данные \\\hline
		Аккаунт & Логин, почта, индекс роли, хэш пароля, имя, описание, изображение \\\hline
		Соревнование & Имя, описания, даты начала и конца, описание уровня, выданы ли награды \\\hline
		Заявка об участии & Количество очков, время участия. Ссылается на соревнование и профиль \\\hline
		Тип награды & Имя, описание, редкость, изображение\\\hline
		Награда & Дата выдачи. Ссылается на тип награды, соревнование и игрока \\\hline
		Награда соревнования & Ссылается на тип награды, соревнование, и критерий выдачи по месту или рангу.\\\hline
		Уровень соревнования & Платформа, версия уровня, файл уровня. Ссылается на соревнование\\\hline
	\end{tabular}
\end{table}
\FloatBarrier
Диаграмма сущностей базы данных представлена на рисунке~\ref{scheme:erd}
\scheme{erd}{Диаграмма "сущность-связь" в нотации Чена}

\FloatBarrier
\section{Категории пользователя}

В рамках задачи было выделено три категории пользователя:
\begin{itemize}
	\item гость,
	\item игрок,
	\item админинстратор.
\end{itemize}

Все категории пользователя имеют возможность просматривать профили игроков, таблицы лидеров соревнования, а также собственно соревнований (данные о сроках проведения и наградах)

Гость имеет возможность авторизации и создания аккаунта. При авторизации гость может стать игроком или администратором.

Игрок может просматривать свои награды, редактировать свой профиль, а также участвовать в соревнованиях.

Админинстратор может:
\begin{itemize}
	\item назначать и отзывать награды игрока;
	\item создавать и редактировать соревнования;
	\item отзывать нежелательные результаты, удаляя их;
	\item создавать и редактировать типы наград.
\end{itemize}

Игрок и администратор имеют возможность выйти из аккаунта, в последствии чего устанавливается категория пользователя "гость".

Диаграмма пользования базой данных представлена на рисунке~\ref{scheme:usecase}.

\scheme{usecase}{Диаграмма пользования базой данных}

\section{Выбор модели данных}

База данных — это документированное собрание интегрируемых записей.
Логическую структуру базы данных определяет её модель. Основными моделями базы данных являются~\cite{karpova}:
\begin{itemize}
	\item иерархическая модель;
	\item cетевая модель;
	\item реляционная модель.
\end{itemize}

Иерархическая модель позволяет строить БД с иерархической древовидной структурой~\cite{karpova}. В иерархической модели данных используется ориентация древовидной
структуры от корня к листьям~---~любой дочерний узел имеет только один родительский узел. Исходя из этой особенности, реализация связей многие ко многим не поддерживается.
Поскольку в рамках задачи присутствует отношение многие ко многим, данная модель не подходит для достижения цели.

Сетевая модель данных основывается на графе зависимости. Она является наиболее полной с точки зрения реализации различных типов связей и ограничений целостности. Но т.к. наборы организованы с помощью физических ссылок, в этой модели не обеспечивается физическая независимость данных: изменение структуры данных требует изменения логики приложения~\cite{karpova}.

Реляционная модель данных и представляет собой набор отношений, изменяющихся во времени. Существуют три части реляцонной базы данных~\cite{relationdb}:
\begin{itemize}
	\item структурная часть описывает, из каких объектов состоит реляционная модель;
	\item целостная часть определяет базовые требования целостности;
	\item манипуляционная часть описывает способы манипулирования данными.
\end{itemize}

Реляционная модель данных основана на понятии отношения~---~информационной модели реального объекта предметной области, формально представленной множеством однотипных кортежей. Кортеж отношения соответствует экземпляру объекта, свойства которого определяются значениями соответствующих атрибутов (полей) кортежа. Кортежи отношений могут быть связаны между собой с помощью внешних ключей~--~ссылок на соответствующие атрибуты.

В качестве модели данных была выбрана реляционная модель, поскольку в ней обеспечена физическая независимость данных и возможно реализовать отношение многие ко многим.

\section*{Вывод}

Соревновательная игра была формализована. Проведён анализ существующих решений. Сущности базы данных были формализованы. Была выбрана реляционная модель данных в силу физической независимости данных, а также возможности реализовать отношение многие ко многим.

\clearpage
